\chapter{関連研究}
\label{chap:kanren}

本章では関連研究を紹介し、それらの特徴や本研究との関連性について示す。

\newpage

本研究では手書きデータをWikiシステムのコンテンツとして活用することを検討したが、手書きデータの活用を目的とする先行研究はWiki以外にも数多く存在する。

\section{主要な研究領域}
手書きメモ・イラストに関する主要な研究領域を解説する。

\section{手書きメモ支援システム}

\section{デザイン・プロトタイピングツール}

\section{手書きデータを利用したコラボレーションツール}

\begin{enumerate}
    \item 手書きメモ支援システム\\
    楽譜上のどの部分を演奏しているのか認識するためのスコアアラインメント技術が研究されている\cite{online}\cite{learning}\cite{coupled}\cite{automatic}。
    応用例として計算機による伴奏システム\cite{muens}\cite{mysong}、音楽鑑賞サポートシステム\cite{orchestra}、楽器練習サポートシステムなどが挙げられる\cite{tutor}。
    \item 手書きデータから高度な\\
    ジェスチャーを利用した楽譜入力インターフェースが提案されている\cite{notepad}\cite{pen}\cite{sssp}。
    これにより、計算機上でペンやポインティングデバイスを使って、楽譜を手書き入力することができる。
    \item 楽譜認識/再生システム\\
    画像から楽譜情報を読み取るための楽譜認識技術が研究されている\cite{optical}\cite{early}\cite{symbol}。
    これを応用して、カメラを搭載したデバイスによって印刷された楽譜を再生するシステムが提案されている\cite{onnote}\cite{gocen}。
\end{enumerate}