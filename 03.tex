\chapter{設計}
\label{chap:sekkei}

本章ではハイパーイラストと手書きベースWikiの要件と設計について述べる。

\newpage

\section{要件}
前章で示した画像ファイルフォーマットや手書きデータを扱う既存のツールの問題点を踏まえて、本システムの要件を整理する。
\begin{enumerate}
    \item 簡単に手書きメモのメモやイラストが作成・編集できる\\
    メモを取るような気軽さで楽譜を書くことができ、新規作成/既存楽譜の編集両方を簡単に行える。
    \item 作成した手書きのメモやイラストを簡単に参照したり、再利用したりできる\\
    楽譜上の要素やテキストに対してハイパーリンクを設定でき、関連情報に素早くアクセスできる。
\end{enumerate}
これらの要件を満たすシステムは次世代の画像フォーマットであるハイパーイラストと、その作成・編集と管理をサポートする手書きベースWikiの組み合わせによって実現可能である。

\section{ハイパーイラスト}
複数の文書間の参照を示すハイパーリンクが埋め込まれた文書はハイパーテキストと定義される。
同様に本研究においてハイパーリンクが埋め込まれた手書きの図やメモをハイパーイラストと定義する。\\
この要件を満たすため、本研究におけるハイパーイラストはSVG\cite{aboutsvg}というフォーマットをベースとしている。
SVGはXML\footnote{https://www.w3.org/XML/}をベースとしており、他の画像ファイルフォーマットにはない、ハイパーイラストに適した以下のような特徴を持つ。
\begin{itemize}
    \item 構造を保持できる\\
    SVGにおいて各ストロークは独立した要素であり、個別に移動や変形等の操作が可能である。またレイヤーや書き順等の構造も保持できるため
    異なるSVGをネストして配置できるためビットマップ画像と比較して再編集・再利用が行いやすい。
    \item Web標準の技術である\\
    SVGは特定の企業の製品ではなく、その仕様は全て公開されている。また表示に特別なソフトウェアを必要とせず、
    ブラウザのみで閲覧することができる。
    \item ハイパーリンクを埋め込める\\
    XLink形式のハイパーリンクを任意の要素に埋め込むことができる。画像の中の個別の要素に対して別々のリンク
    を定義することができる。
\end{itemize}

\section{手書きベースWiki}
