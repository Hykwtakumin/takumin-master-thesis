\chapter{設計}
\label{chap:sekkei}

本章ではハイパーイラストと手書きベースWikiの要件と設計について述べる。

\newpage

\section{要件}
前章で示した画像ファイルフォーマットや手書きデータを扱う既存のツールの問題点を踏まえて、本システムの要件を整理する。
\begin{enumerate}
    \item 簡単に手書きメモのメモやイラストが作成・編集できる\\
    メモを取るような気軽さで楽譜を書くことができ、新規作成/既存楽譜の編集両方を簡単に行える。
    \item メモなどの情報を自在に書ける\\
    テキストで楽譜を記述でき、ハイパーテキストのようなフォーマットの中で楽譜以外の情報も一緒に管理できる。
    \item 作成した手書きのメモやイラストを簡単に参照したり、再利用したりできる\\
    楽譜上の要素やテキストに対してハイパーリンクを設定でき、関連情報に素早くアクセスできる。
\end{enumerate}
これらの要件を満たすシステムは次世代の画像フォーマットであるハイパーイラストと、その作成・編集と管理をサポートする手書きベースWikiの組み合わせによって実現可能である。

\section{ハイパーイラスト}


\section{手書きベースWiki}
