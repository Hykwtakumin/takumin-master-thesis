\chapter{序論}
\label{chap:introduction}

本章では本研究の動機と目的、および本論文の構成について述べる。

\newpage

\section{研究の動機}

手書きでメモを取ったりイラストを描いたりすることは、情報を記録し、表現する手段として一般的であるが、
その様式は紙や鉛筆等の筆記具が発明された頃からほとんど変わっておらず、一枚の紙の上で表現する事を前提としているため
参照や管理、再利用が難しいという問題点が存在する。

また計算機が普及した現在では、手書きメモ・イラストを紙の上ではなくデータとして作成するソフトウェアも広く利用されているが、
それらはPNG\footnote{http://www.libpng.org/pub/png/}や JPEG\footnote{https://jpeg.org/jpeg/}のような
一枚の絵をピクセルの集合で単純に置き換えた形式で記録されることが一般的で、紙に描かれたものと比較して本質的に変化していない。
この制約により計算機上で作成したメモやイラストであっても、すぐに参照できるようにするためには階層化やタグ付け等の運用上の工夫が要求され、
紙の上で手書きのメモを取っていた時と同じ不便さを引き継いでいる。

一方で手書きメモやイラストと同じく紙の上で記録されていたテキストは、計算機の登場により以下のように変化した。
\begin{itemize}
    \item 他の文書への参照を実現するハイパーリンクと、それを内包した文書であるハイパーテキストが登場した
    \item Webによって様々なハイパーテキストに手軽にアクセスできるようになった
    \item コラボレーションツールであるWikiが複数人による共同編集を可能にし、知見の共有を実現した
\end{itemize}

これにより参照や管理・再利用が難しいという問題が解決された。
かつては手書きメモ・イラストと同様の問題を抱えていたテキストは、計算機による新しい活用法が発明された事で広く普及するに至った。
そのため手書きメモ・イラストも、計算機を活用する事で問題を解決し、進化する余地があると考えられる。

\section{研究の目的}
%本研究では、手書きメモ・イラストを表現する既存のフォーマットが抱える問題を解決し、またハイパーテキストやWiki等の技術を取り入れることで
%従来のシステムでは実現できなかった手書きデータの参照・管理・再利用環境を実現するシステム「手書きベースWiki」の構築を目的とする。
本研究では、手書きメモ・イラストを扱う既存のシステムが抱える参照や管理、再利用の難しさといった問題を解決し、またハイパーテキストやWiki等の技術を取り入れることで
従来のシステムでは実現できなかった手書きメモ・イラストの新しい活用法を実現するシステム「手書きベースWiki」の構築を目的とする。

\newpage

\section{本論文の構成}

本論文は以下の8章で構成される。

第2章では、本研究の背景をより詳細に分析し、既存システムの問題点を整理する。

第3章では、本論文で提案するシステムの基本構成と使い方について述べる。

第4章では、本論文で提案するシステムの詳細な実装について述べる。

第5章では、本論文で提案するシステムによって実現可能な応用例について述べる。

第6章では、関連する研究を紹介し、それらの特徴や本研究との関連を述べる。

第7章では、筆者による運用経験やユーザーからのフィードバックをまとめ、本論文で提案するシステムの有効性と問題点について述べる。

最後に、第8章で本論文のまとめと結論を述べる。