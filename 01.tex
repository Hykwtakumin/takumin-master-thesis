\chapter{序論}
\label{chap:introduction}

本章では本研究の動機と目的、および本論文の構成について述べる。

\newpage

\section{研究の動機}
\label{douki}


手書きによってメモやイラストを描くことは情報を記録し表現する手段として一般的である。
%人間は昔から紙とペンで情報を記録してきた
ペンによるインターフェースを備えたタブレットデバイスが普及した現在では計算機上でも手書きメモやイラストを描くようになったが、
その運用は紙の上で記録していた時代から進歩がなく、不便な点が解決されていないのが現状である。
例えば描いたメモやイラストそのものを検索することができないため参照や管理が難しく、
また引用する仕組みもないので再利用が難しいという問題が存在する。
%紙に描いた情報は検索も再利用もできないので不便だった
%しかしテキストに関してはそのような問題は解決している

%先んじてというのは変な表現
一方で手書きメモに先んじて計算機上で扱われるようになったテキストは、検索が可能なため
参照や管理が行いやすく、文書作成やメール等の様々な用途で積極的に利用されるようになった。
さらにWebやハイパーリンクによって目的の文書を参照したり、引用することができるようになり、
またハイパーリンクを含んだ文書(ハイパーテキスト)を手軽に作成・編集できるWikiシステムによって
より活発・柔軟な情報の再利用が実現された。
したがって同様の技術を手書きのメモやイラストに適用することで、参照や管理、再利用が難しいという問題点を解決することができると考えられる。
例えば手書きメモをハイパーテキストのように扱うシステムがあれば、メモ間のリンクに基づいた検索を行うことが可能なため参照や管理がしやすくなる。
またリンクによって別のメモやイラストを引用するといった再利用も可能になる。
\\
本研究ではハイパーテキストのような手書きのメモやイラストを自在に作成し、Wikiのように運用できるシステム「手書きベースWiki」を開発することで
手書きメモが抱える問題の解決を解決した。

%手書きによってメモやイラストを描くことは情報を記録し表現する手段として一般的である。
%ペンによるインターフェースを備えたタブレットデバイスが普及した現在では計算機上でも手書きメモやイラストを描くようになったが、
%その様式は紙の上で記録していた頃と比較して大きく変わっていない。
%%参照や管理が難しいという主張
%%内容によって検索ができないことから編集可能になったということぐらいしか便利になっていない
%例えば手書きメモを参照するためには適切なフォルダに分類したり タグやラベルを付けて管理する等の運用上の工夫が求められるが、
%これは紙に描いたメモを物理的に整理するのと同じ不便さを再現してしまっている。
%%再利用が難しいという主張
%また整理された手書きメモは互いに独立した一枚絵として扱われているため、メモの内部に別のメモを引用するといった使い方はできない。
%すなわち1.参照や管理が面倒、2.再利用が難しい という紙に記録されていた手書きメモが抱える問題点が解決されずにいるのが現状である。
%\\
%一方かつては手書きメモと同様に紙の上で記述されていたテキストも同じ問題を抱えていたものの、1.ハイパーリンクによって管理に気を
%配らずとも目的の文書を参照できるようになり、2.Wikiシステムによってリンクを含んだ文書を手軽に作成・編集可能になったことで
%さらなる知見の共有や情報の再利用が実現された。
%したがって同様の技術を手書きメモに対しても適用することで問題点を解決することができると考えられる。
%例えば手書きメモの内部に別のメモへのリンクが貼ることができれば情報の再利用が実現され、
%さらにリンクに基づいて目的のメモが参照できれば管理の不便さも解消できる。
%また手書きメモをコンテンツとして扱うWikiシステムは、既存のメモアプリケーションや
%テキストベースのWikiよりも自由度の高いものとなるだろう。
%\\
%そこで本研究では手書きのメモやイラストにハイパーリンクを埋め込み、Wikiとして活用できるシステムを実装し、
%これらの問題を解決することを目的とする。
%
%%%現状手書きはどういう状態なのか?どのような問題があるのかを整理せよ
%%手書きメモにハイパーリンク要素を入れると、何が解決して何が嬉しいのか?どう改善されるのか例も示す

\section{研究の目的}
%なぜ解決されるのか?というのがわかるように前の節を書くべし
本研究では手書きのメモをWikiとして扱えるシステム「手書きベースWiki」を開発し、第\ref{douki}節で述べた手書きメモ・イラストが抱える問題
を解決することを目的とする。またシステムの利用例を示し、その有用性を検証する。

%本研究では第\ref{douki}節で述べた手書きメモ・イラストが抱える問題点を共有していたテキストを大きく改善したハイパーテキストやWikiのような
%技術に着目し、それらを手書きメモ・イラストに応用することで問題解決を実現する「手書きベースWiki」を構築する。
%また手書きベースWikiによって実現可能な応用例を提案し、その有用性を検証する。

\newpage

\section{本論文の構成}

本論文は以下の8章で構成される。

第\ref{chap:haikei}章では、本研究の背景をより詳細に分析し、既存システムの問題点を整理する。

第\ref{chap:sekkei}章では、本論文で提案するシステムの基本構成と使い方について述べる。

第\ref{chap:zissou}章では、本論文で提案するシステムの詳細な実装について述べる。

第\ref{chap:riyourei}章では、本論文で提案するシステムの利用例を紹介する。

第\ref{chap:kanren}章では、関連する研究を紹介し、それらの特徴や本研究との関連を述べる。

第\ref{chap:kosatsu}章では、筆者による運用経験やユーザーからのフィードバックをまとめ、本論文で提案するシステムの有効性と問題点について述べる。

最後に、第\ref{chap:ketsuron}章で本論文のまとめと結論を述べる。