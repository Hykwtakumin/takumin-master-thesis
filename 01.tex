\chapter{序論}
\label{chap:introduction}

本章では本研究の動機と目的、および本論文の構成について述べる。

\newpage

\section{研究の動機}
\label{douki}

手書きによってメモやイラスト描くことは情報を記録し表現する手段として一般的である。
%人間は昔から紙とペンで情報を記録してきた
ペンによるインターフェースを備えたタブレットデバイスが普及した現在では計算機上でも手書きメモやイラストを描くようになったが、
描いたメモやイラストそのものを検索することができないため参照や管理が難しく、
また引用する仕組みもないので再利用が難しいという問題が解決されていないのが現状である。

一方でテキストデータは、検索が可能なため参照や管理が行いやすく、文書作成やメール等の様々な用途で積極的に利用されている。
さらにWebやハイパーリンクによって目的の文書を参照したり、引用することができるようになり、
またハイパーリンクを含んだ文書(ハイパーテキスト)を手軽に作成・編集できるWikiシステムによって
より活発・柔軟な情報の再利用が実現された。
したがって同様の技術を手書きのメモやイラストに適用することで、参照や管理、再利用が難しいという問題点を解決することができると考えられる。
例えば手書きメモをハイパーテキストのように扱うシステムがあれば、メモ間のリンクに基づいた検索を行うことが可能なため参照や管理がしやすくなる。
またリンクによって別のメモやイラストを引用するといった再利用も可能になる。
\\
本研究ではハイパーテキストのような手書きのメモやイラストを自在に作成し、Wikiのように運用できるシステム「手書きベースWiki」を開発することで
手書きメモが抱える問題を解決した。

\section{研究の目的}
本研究では手書きのメモをWikiとして扱えるシステム「手書きベースWiki」を開発し、第\ref{douki}節で述べた手書きメモ・イラストが抱える問題
を解決することを目的とする。またシステムの利用例を示し、その有用性を検証する。

\newpage

\section{本論文の構成}

本論文は以下の8章で構成される。

第\ref{chap:haikei}章では、本研究の背景をより詳細に分析し、既存システムの問題点を整理する。

第\ref{chap:sekkei}章では、本論文で提案するシステムの基本構成と使い方について述べる。

第\ref{chap:zissou}章では、本論文で提案するシステムの詳細な実装について述べる。

第\ref{chap:riyourei}章では、本論文で提案するシステムの利用例を紹介する。

第\ref{chap:kanren}章では、関連する研究を紹介し、それらの特徴や本研究との関連を述べる。

第\ref{chap:kosatsu}章では、筆者による運用経験やユーザーからのフィードバックをまとめ、本論文で提案するシステムの有効性と問題点について述べる。

最後に、第\ref{chap:ketsuron}章で本論文のまとめと結論を述べる。