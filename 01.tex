\chapter{序論}
\label{chap:introduction}

本章では本研究の動機と目的、および本論文の構成について述べる。

\newpage

\section{研究の動機}

手書きでメモを取ったり絵や図を描いたりすることは、情報の記録し、表現する手段として一般的であるが、
その様式は紙や鉛筆等の筆記具が発明された頃からほとんど変わっておらず、一枚の紙の上で表現する事を前提としているため
参照や管理、再利用が難しいという問題点が存在する。

また計算機が普及した現在では、手書きメモ・イラストを紙の上ではなくデータとして作成・記録し、
それらを閲覧・管理できるソフトウェアも広く利用されているが、そのデータはJPEG\footnote{http://www.libpng.org/pub/png/}や
PNG\footnote{https://jpeg.org/jpeg/}のような一枚の絵をピクセルの集合で単純に置き換えた形式で
作成・記録されることが一般的で、紙に描かれたものと比較して本質的に変化していない。
そのため、データのタイムスタンプを元に時系列で管理したり、あるいは事前に取り決めた場所(ディレクトリ)に保存する等の工夫が必要とされ、
紙の上で手書きのメモを取っていた時と同質の不便さを引き継いでいる。

一方でメモやイラストと同じく紙の上で記録されていたテキストは、計算機の登場により以下のような機能を獲得した。

\begin{itemize}
    \item コピー・ペースト/Undo・Redo等の編集支援機能
    \item ハイパーリンクとWebによる他のテキスト(ドキュメント)への参照
    \item 複数人による共同編集を可能にするWiki等のコラボレーションツール
\end{itemize}

これにより(1)簡単に編集できない、(2)参照や管理が面倒という問題が解決された。
かつては手書きメモ・イラストと同様の問題を抱えていたテキストは、計算機による新しい活用法が発明された事で広く普及するに至った。
そのため手書きメモ・イラストの活用も、計算機を活用する事で同様の進化が可能であると考えられる。

\section{研究の目的}
本研究では、手書きメモ・イラストを表現する既存のフォーマットが抱える問題を解決し、またハイパーテキストやWikiの手法を取り入れることで、
従来のシステムでは実現できなかった手書きデータの参照・管理・再利用環境を実現するシステム「手書きベースWiki」の構築を目的とする。

\newpage

\section{本論文の構成}

本論文は以下の8章で構成される。

第2章では、本研究の背景をより詳細に分析し、既存システムの問題点を整理する。

第3章では、本論文で提案するシステムの基本構成と使い方について述べる。

第4章では、本論文で提案するシステムの詳細な実装について述べる。

第5章では、本論文で提案するシステムによって実現可能な応用例について述べる。

第6章では、関連する研究を紹介し、それらの特徴や本研究との関連を述べる。

第7章では、筆者による運用経験やユーザーからのフィードバックをまとめ、本論文で提案するシステムの有効性と問題点について述べる。

最後に、第8章で本論文のまとめと結論を述べる。