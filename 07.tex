\chapter{考察}
\label{chap:kosatsu}

本章では、手書きベースWikiの利用者の意見や自身の運用経験をまとめ、また関連研究を踏まえシステムの有用性を評価し、諸問題や研究の重要性や発展性について述べる。

\newpage

\section{評価}
本研究によって実装した手書きベースWikiを、以下の項目について評価する。

\begin{enumerate}
    \item 第\ref{mondai}節で述べた手書きメモ・イラストの問題点を解決できたか
    \begin{itemize}
              \item 参照や管理が面倒
              \item 再利用が難しい
    \end{itemize}
    \item 手書きメモの内部にリンクがあることの有用性
    \item 手書きメモ・イラストがWiki的に扱えることの有用性
\end{enumerate}


手書きベースWikiのプロトタイプとなる「DrawWiki」は2019年10月にWebアプリケーションとして公開した\footnote{ \textsf{https://draw-wiki.herokuapp.com/} }。
またORF2019\footnote{ \textsf{https://orf.sfc.keio.ac.jp/2019/} }にてDrawWikiの展示発表を行った。
本節では、DrawWikiの展示発表で得られた感想、ユーザーからのフィードバックや筆者の運用経験を元にシステムを評価する。

\subsection{意見・感想}

第\ref{mondai}節で述べた手書きメモ・イラストの問題点や、その解決策として
手書きメモ・イラストをハイパーテキストとして扱い、Wiki的に運用する本システム
について多くの同意が得られた他、以下のような感想や意見が寄せられた。

\paragraph*{関連資料のナレッジ化}
デザインの課題では多数のスケッチを描く場面があるが、それらは書き捨てが前提の紙で管理されているため
管理しなければ散逸してしまい、また振り返りのために参照するのも大変である。
これを既存のメモアプリケーションに置き換えても紙と同様の参照・管理の不便さや再利用の難しさは解決しないが、
手書きベースWikiであればスケッチにリンクを追加することで関連する作品を漏れなく参照することができ、
生きたナレッジとして活用できそうであるという意見があった。

%Wikiへの参加のしやすさの向上とかをうまく言い表したい
\paragraph*{Wikiへのアクセシビリティの向上}
\label{tegakiuser}
タイピングやPCの操作が得意でないために既存のテキストベースのWikiを使う機会がなかった人からも、
手書きのメモやイラストがリンクによって繋がる仕組みがあれば自分でもコンテンツを作りWikiのように
運用ができそうという意見があった。

\subsection{筆者の運用経験}
主に個人用のメモとして活用しつつ、DrawWikiの開発と並行して3ヶ月間利用した。

\paragraph*{リンクに基づく優れた参照性}
DrawWikiにはフォルダやタグ・ラベル等のメモやイラストを分類し管理するための機能は存在しない。
全てのメモ・イラストがフラットに置かれているものの、リンク情報に基づいて関連するメモ・イラストが推薦されるため、参照に困ることはなかった。
既存のメモアプリではメモを描く際、これはどのフォルダに分類されるべきか、どのタグやラベルをつけるべきか
といったことを意識していたが、管理に特別気を配らなくてもリンクによるナビゲーションがあれば目的のメモ・イラストを
参照できることがわかった。

\paragraph*{リンクを持つ手書きメモの有用性}
第\ref{chap:riyourei}章で示したように、ソフトウェアのプロトタイプやインタラクティブな図解等、従来の手書きメモ・イラストを扱う
アプリケーションでは表現できなかったコンテンツをDrawWikiの機能によって作成することができた。
またSVGの仕様により画像本体が他のメモ・イラストへのリンクを保持できるため、関連情報を欠落させることなく
DrawWiki以外のWebサイトでも画像を埋め込み、再利用できるようになった。


\subsection{問題点・要望}
以下のような問題点が明らかになった。
\begin{enumerate}
    \item テキスト検索によってメモを参照したい\\
    メモの内部にある手書きの文字を対象に、テキスト検索によって参照できる機能がほしいという要望があった。
    現状のDrawWikiは文字として描かれた手書きストロークを特別に区別・認識していないため、テキストによって検索する機能を備えていない。
    \\
    DrawWikiではスタイラスから得られる手書きデータを元にストロークを生成しているため、オンライン手書き文字認識を応用することで
    メモ内のテキストを認識し、 クエリとして検索・参照する機能は技術的に実装可能である。
    テキスト検索機能を備えることでよりメモを参照しやすくなると考えられる。
    \item 編集履歴が分かりづらい\\
    DrawWikiで作成されたメモ・イラストはURLを通じて編集可能であるが、どのように編集されたのか、どのような内容が追記されたのかを
    判断しづらい場合がある。また誰によって編集されたのかを視覚的に表示する機能を備えていないため、共同編集を行った際に
    どの変更が誰によるものなのかを把握しづらいという指摘があった。
    \\
    DrawWikiではファイル内部に編集に関する履歴情報を保持しているためこれらの情報を元に
    編集者によってストロークの色や形状に変更を加えたり、新しく編集された部分について強調表示したり
    等の視覚表現を追加する機能があればよりわかりやすく共同編集を行うことが可能になると考えられる。
    \item リンクを追加する際のサジェストが欲しい\\
    DrawWikiではある要素に対して別のメモ・イラストをリンクさせる際に、今までに作成したメモ・イラストの一覧から選択するという操作を行う。
    この一覧は時系列順や引用数/被引用数によってソート・絞り込みを行うことが可能だが、さらに進んでメモを取っている最中にその時点での手書きストロークに類似する
    過去のメモやイラストを候補として推薦する仕組みがあれば、よりスムーズなリンクの追加が可能になるという意見があった。
    \\
    スケッチしている図形を認識したり\cite{Notowidigdo2004OfflineSI}\cite{Wobbrock2007GesturesWL}、
    それをクエリとして画像検索を行う手法は数多く存在する\cite{Eitz2012SketchbasedSR}\cite{Eitz2011PhotosketcherIS}
    \cite{Chen2009Sketch2PhotoII}。これらを応用することで描画中の手書きストロークを 元に過去に作成したメモ・イラストの中から
    類似したものを検索し、候補として提案する"手書き入力補完"のような機能を実装することができ、より
    スムーズに手書きメモ・イラスト同士をリンクさせることができるようになる。
\end{enumerate}

\section{考察}


\subsection{手書きメモ・イラストの問題点の検証}
本システムにおいて第\ref{mondai}節で述べた問題点が克服されているかどうかを検証する。
\begin{itemize}
    \item 参照や管理が面倒\\
    手書きメモ・イラストのリンク情報に基づいて関連画像を推薦・表示する機能により、
    目的のメモ・イラストを参照できるようになった。
    \item 再利用が難しい\\
    手書きメモやイラストに対して別のメモを引用したり、リンク情報を保持したまま他のWebサイトに埋め込んだりといった
    再利用が可能になった。
\end{itemize}
以上のように、第\ref{mondai}節で述べた問題点は手書きベースWikiによって解決することができた。

\subsection{手書きメモの内部にリンクがあることの有用性}
第\ref{chap:riyourei}章で述べた利用例は、手書きメモの内部の任意の要素にリンクを保持し、
ダイアログ表示するという機能によって実現できたものである。また第\ref{sketchcomm}節でも示されている通り
図の中に注釈や別の図を埋め込めることによって表現力が高まることは明らかである。
そして内部にリンクを持った画像を作成しさらにWikiとして運用することは第\ref{hyperillustcreators}節で挙げたような既存のツールでは
困難な作業である。簡単な操作で内部にリンクを持った手書きメモ・イラストを作成し、
Wiki的に運用する手段を手書きベースWikiによって初めて提供することができたと言える。

\subsection{手書きメモをWiki的に使えることの有用性}
Wikiは有用なソフトウェアであるが、ある程度の速度でテキスト入力ができることや、
コンテンツを記述するために記法を習得することをユーザーに求めているため敷居が高い。
手書きベースWikiはテキスト入力の必要もなく、記法も存在しないため
第\ref{tegakiuser}節で述べたようなユーザーでも参加できるという利点がある。
手書きというユニバーサルな表現手段でコンテンツを作り、ナレッジとして運用可能にした手書きベースWikiは
Wikiのアクセシビリティを向上させた。

\section{まとめ}
手書きベースWikiは第\ref{mondai}節で述べた問題点を解決した。
また手書きメモ・イラストの内部にリンクを埋めこむ機能によって、優れた表現力を持つ有用な利用例を示した。
さらにアクセシビリティの面でテキストベースWikiに対しても優位性があり、より多くの人々にとって
使えるWikiになりえると考えられる。