\chapter{考察}
\label{chap:kosatsu}

本章では、手書きベースWikiの自身の運用経験や利用者の意見をまとめ、諸問題や研究の重要性・発展性について述べる。

\newpage

\section{評価}
手書きベースWikiのプロトタイプとなる「DrawWiki」は2019年10月にWebアプリケーションとして公開した\footnote{ \textsf{https://draw-wiki.herokuapp.com/} }。
またORF2019\footnote{ \textsf{https://orf.sfc.keio.ac.jp/2019/} }にてDrawWikiの展示発表を行った。
本節では、DrawWikiの筆者の運用経験や展示発表で得られた感想、ユーザーからのフィードバックをまとめる。

\subsection{筆者の運用経験}
主にxxx/xxx/xxxの用途でDrawWikiを利用した。

\paragraph*{個人のメモとして}
%筆者はクラシックギターの練習に本システムを利用しており、練習中に残した様々なメモを整理して記録できている。
%以前まではたくさん書き込んでも練習する曲それぞれの楽譜に散逸してしまうため整理や再利用が難しく、結局役に立たないという問題に悩まされていたが、本システムによって解決された。
\paragraph*{手書きのWikiとして}
%既存の教材に強く苦手意識を抱いていたが、自身が演奏している楽曲と知識の関係性がハイパーリンクによって可視化され、自分専用の分かりやすい教材として利用できている。
%また音声の埋め込みによって実際の演奏を聞きながら学習でき、イメージしづらい概念の理解を深めるのに特に有効である。
\paragraph*{その他}
%筆者は本システムの利用前までABCの利用経験が無かったが、複雑な楽譜でない限りは楽譜IMEの支援によって快適に楽譜編集できている。
%また採譜した楽曲の音声/動画/作曲者/演奏者といった周辺情報を一元管理でき便利である。

%\subsection{ユーザーインタビュー}
%%予備調査を行ったことを書く?書くなら付録も用意すべき


\subsection{意見・感想}
%以下のような意見が得られた

\paragraph*{授業の資料として}
\paragraph*{関連資料をまとめる手段として}
デザインの課題では多数のスケッチを描くことがあるが、それらは紙で管理されているため
散逸してしまう
(紙だと参照しづらいという問題がまず解決した)
(Wikiという手法によってナレッジ化ができるようになった)
    (そもそもWikiを個人で使ったことはなかったが、それはとっつきにくかったから)
    (手書きというカジュアルな方法で記述できるようになったので使えるようになった)

\subsection{問題点・要望}
以下のような問題点が明らかになった。
\begin{enumerate}
    \item IMEがほしい\\
%    複数の音符にハイパーリンクが設定されている場合、すべての音符が1つのリンクを示しているのか、それぞれの音符が別のリンクを示しているのか一目見て理解できないという意見が挙げられた。
%    これはリンク先にかかわらず全て青色の音符として表示しているためであるが、単音に対しても、フレーズや小節に対してもリンクを設定したいので、何らかのインタフェース的工夫が必要である。
    他のメモへのハイパーリンクを埋め込む際に、どのメモを埋め込むかはモーダルによって選ぶ
    \item テキスト検索によってメモを参照したい\\
%    既存の楽譜を取り込めるような仕組みがほしいという意見が挙げられた。
%    楽譜を自ら編集することを前提としているため、既存の楽曲を利用したい場合は元になる楽譜を別途用意したり、採譜したりして自分で入力する必要がある。
%    楽譜IMEによってある程度入力作業が簡単になるものの、すぐ楽譜を利用したいユーザーにとっては大きな負担になる。
    メモの内部にある手書きの文字を対象に、テキスト検索によって参照できる機能がほしいという要望があった。
    現状のDrawWikiにおいて文字として描かれた手書きストロークを特別に区別・認識していないため、テキストによって検索する機能を備えていない。
    テキストによって
    \item 編集履歴が分かりづらい\\
    基本的にハイパーイラストはURLを通じていつでも、誰でも編集可能であるが、どのように編集されたのか、どのような内容が追記されたのかを
    判断しづらい場合がある。また手書きストロークが誰によって追加されたのかを視覚的に表示する機能を備えていないため、共同編集を行った際に
    誰による編集なのかを把握しづらいという指摘があった。
\end{enumerate}

\section{考察}

\subsection{設計の妥当性}
手書きメモ・イラストを相互参照可能にする

\subsection{解決すべき課題}
\begin{enumerate}
    \item 手書きIMEによるリンク追加処理の支援\\
    描画中の手書きストロークをクエリとして検索し、類似する手書きメモを"リンク追加候補"として表示する"手書きIME"のような編集支援機能があれば、
    \item テキスト検索による参照\\
    DrawWikiではスタイラスから得られる手書きデータを元にストロークを生成しているため、オンライン手書き文字認識を応用することで
    メモ内のテキストを認識し、 クエリとして検索・参照する機能は技術的に実装可能である。
    テキスト検索機能を備えることで、よりメモを参照しやすい状態で運用することができると考えられる。
    \item 編集履歴の可視化\\
    ハイパーイラストではストローク等の編集履歴を保持できるフォーマットである。
    これらの情報を元に、
\end{enumerate}

\subsection{手書きメモ・イラストの問題点の検証}

