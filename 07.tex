\chapter{考察}
\label{chap:kosatsu}

本章では、手書きベースWikiの自身の運用経験や利用者の意見をまとめ、諸問題や研究の重要性・発展性について述べる。

\newpage

\section{評価}
手書きベースWikiのプロトタイプとなる「DrawWiki」は2019年10月にWebアプリケーションとして公開した\footnote{ \textsf{https://draw-wiki.herokuapp.com/} }。
またORF2019\footnote{ \textsf{https://orf.sfc.keio.ac.jp/2019/} }にてDrawWikiの展示発表を行った。
本節では、DrawWikiの筆者の運用経験や展示発表で得られた感想、ユーザーからのフィードバックをまとめる。

\subsection{筆者の運用経験}
主に個人用のメモとして活用しつつ、Drawwikiの開発と並行して3ヶ月間利用した。

%絵の中の一部に別の絵を埋めこむ仕組みによって便利ではハイパーな手書きメモが作れるようになったこと
%それによってフラットに、カジュアルに、それでいてちゃんと見つかる繋がる手書きメモのナレッジができたというところ
%これらは既存のツールではできなかったことである。

\paragraph*{ハイパーイラストの有用性}
第\ref{chap:ouyou}章で示したように、ソフトウェアのモックアップや自作の整備メモ等の従来のメモアプリでは表現できなかった
%イラストの内部にリンクを埋め込める
ハイパーイラストならではのコンテンツを作成することができた。
ハイパーイラストの作成そのものはInkscape\footnote{https://inkscape.org/}等のSVGを編集できるツールを用いれば可能であるが、
\begin{itemize}
    \item 手書きの画像データを用意し
    \item 紐付けたい他の画像データをクラウドストレージにアップロードし
    \item そのURLを埋め込んだ上でSVGとして保存する
\end{itemize}
という一連の作業を全て手作業で行う必要があり、気軽に作成することはできなかった。
DrawWikiによって簡単な操作でイラストの内部へのリンクの追加が可能になったため、手書きメモというカジュアルな用途でも
ハイパーイラストを作成し、その恩恵を受けられるようになった。

\paragraph*{リンクに基づく優れた参照性}
DrawWikiで作成したメモ・イラストにはリンクこそ埋め込まれているものの、フォルダやタグ・ラベル等のメモを分類し管理するための機能は存在しない。
全てのメモ・イラストがフラットに置かれているものの、リンク情報に基づいて関連するメモ・イラストが推薦されるため、参照に困ることはなかった。
既存のメモアプリではメモを描く前に
\begin{itemize}
    \item どのようなファイル名にするか?
    \item どのようなフォルダに分類するのか?
    \item どのようなタグ・ラベルをつけるのか?
\end{itemize}
等を意識する必要があったが、そのような管理の工夫をせずとも、リンクによるシンプルなナビゲーションがあれば目的のメモ・イラストを
参照できることがわかった。

\subsection{意見・感想}
%以下のような意見が得られた
第\ref{mondai}節で述べた手書きメモ・イラストの問題点について多くの同意が得られた。
多くのユーザーが同様の問題を抱えつつも、本質的に解決する手段・方法が無かったため
各々が運用上の工夫・苦労を強いられていた。
その他に以下のような感想や意見が寄せられた。

\paragraph*{関連資料のナレッジ化}
デザインの課題では多数のスケッチを描く場面があるが、それらは紙で管理されているため
管理しなければ散逸してしまい、また振り返りのために参照するのも大変である。
これを既存のメモアプリケーションに置き換えても紙と同様の参照・管理の不便さが
解決しないが、スケッチにリンクを追加することで関連する作品を漏れなく参照することができ、
生きたナレッジとして活用できるのでは無いかという意見を得た。

\paragraph*{手書きで作成できるWiki}
タイピングやPCの操作が得意でないためにテキストベースWikiを使う機会がなかった人からも、
手書きのメモを描いてそれが繋がるのであれば自分でもコンテンツを作れそう
という意見があった。
第\ref{tegakiwiki}節で示したように、手軽にハイパーイラストが作成でき、
それらをコンテンツとするWikiがあれば、高度なスキルがなくてもカジュアルに
Wikiに参加することが可能であると考えられる。

\subsection{問題点・要望}
以下のような問題点が明らかになった。
\begin{enumerate}
    \item テキスト検索によってメモを参照したい\\
    メモの内部にある手書きの文字を対象に、テキスト検索によって参照できる機能がほしいという要望があった。
    現状のDrawWikiにおいて文字として描かれた手書きストロークを特別に区別・認識していないため、テキストによって検索する機能を備えていない。
    手書きメモであってもテキストによって全文検索したい
    \item 編集履歴が分かりづらい\\
    基本的にハイパーイラストはURLを通じていつでも、誰でも編集可能であるが、どのように編集されたのか、どのような内容が追記されたのかを
    判断しづらい場合がある。また手書きストロークが誰によって追加されたのかを視覚的に表示する機能を備えていないため、共同編集を行った際に
    誰による編集なのかを把握しづらいという指摘があった。
    \item リンクを追加する際の補助が欲しい\\
    他のメモへのハイパーリンクを埋め込む際に、どのメモを埋め込むかはモーダルによって選ぶ
    DrawWikiではある要素に対して別のメモ・イラストをリンクさせたる際に、今までに作成したメモ・イラストの一覧から選択するという操作になっている。
    この一覧は時系列順や引用数/被引用数によってソート・絞り込みを行うことが可能だが、さらに進んでメモを取っている最中にその時点での手書きストロークに類似する
    過去のメモやイラストを候補として推薦する仕組みがあれば、よりスムーズなリンクの埋めこみやイラストのインポートが可能になるという意見があった。
\end{enumerate}

\section{考察}

\subsection{設計指針の妥当性}
本システムは既存のメモアプリケーションとは異なる利用形態を持つが、実際に利用した、またはデモを体験したユーザーからは概ね
好意的に受け止められたため、ハイパーリンクとWikiの組み合わせという設計は適切であったと言える。
本システムをベースに様々な改良を重ねることで、より優れた手書きメモ・イラストの利用環境の実現や応用が可能であると考えられる。

\subsection{解決すべき課題}
\begin{enumerate}
    \item テキスト検索による参照\\
    DrawWikiではスタイラスから得られる手書きデータを元にストロークを生成しているため、オンライン手書き文字認識を応用することで
    メモ内のテキストを認識し、 クエリとして検索・参照する機能は技術的に実装可能である。
    テキスト検索機能を備えることで、よりメモを参照しやすい状態で運用することができると考えられる。
    \item 編集履歴の可視化\\
    DrawWikiで管理されるハイパーイラストでは編集に関する履歴情報を保持している。
    これらの情報を元に、編集者によってストロークの色や形状に変更を加えたり、新しく編集された部分について強調表示したり
    等の視覚表現を追加する機能があればより便利に共同編集を行うことが可能になると考えられる。
    \item "手書き入力補完"によるリンク追加処理の支援\\
    スケッチしている図形を認識したり\cite{Notowidigdo2004OfflineSI}\cite{Wobbrock2007GesturesWL}、
    それをクエリとして画像検索を行う手法は数多く存在する\cite{Eitz2012SketchbasedSR}\cite{Eitz2011PhotosketcherIS}
    \cite{Chen2009Sketch2PhotoII}。これらを応用することで描画中の手書きストロークを 元に過去に作成したメモ・イラストの中から
    類似したものを検索し、候補として提案する"手書き入力補完"のような機能を実装することができ、より
    スムーズに手書きメモ・イラスト同士をリンクさせることができるようになる。
\end{enumerate}

\subsection{手書きメモ・イラストの問題点の検証}
本システムにおいて第\ref{mondai}節で述べた問題点が克服されているかどうかを検証する。
\begin{itemize}
    \item 参照や管理が面倒\\
    手書きメモ・イラストのリンク情報に基づいて関連イラストを推薦する機能により、
    管理に労力を割くことなく目的のメモ・イラストを参照できるようになった。
    \item 再利用が難しい\\
    手書きメモやイラスト同士をリンクさせたり、他のWebサイトに埋め込んだりといった
    再利用が可能になった。
\end{itemize}
以上のように、第\ref{mondai}節で述べた問題点がハイパーイラストと手書きベースWikiによって解決した。