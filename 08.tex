\chapter{結論}
\label{chap:kekka}

本章では本研究を総括する。

\newpage

\section{研究の成果}
本研究では、ハイパーリンクを内蔵した次世代の手書きデータ記述形式「ハイパーイラスト」と、それをコンテンツとして扱うWikiシステム「手書きベースWiki」の提案を行った。
まず第\ref{chap:haikei}章において、既存の手書きメモ・イラストの問題点をテキストの進化と比較しながら分析した。既存システムの現状をとりあげ、計算機が普及した現在も根本的に解決されていないことを示した。
第\ref{chap:sekkei}章では、第\ref{chap:haikei}章で述べた手書きメモの問題点に対する解決方法を提案した。また、それに基づき本研究で開発した「ハイパーイラスト」「手書きベースWiki」の基本構成と使い方について述べた。
第\ref{chap:jissou}章では、「手書きベースWiki」のアプリケーション構成と詳細な実装について述べた。
第\ref{chap:ouyou}章では、「手書きベースWiki」によって実現可能な応用例について述べた。
第\ref{chap:kanren}章では、本研究に関連する研究を紹介し、それぞれのアプローチの特徴と問題点を分析した。
第\ref{chap:kosatsu}章では、筆者による運用経験やユーザーからのフィードバックをもとに本研究の有効性と問題点を分析した。

\section{総括}
本研究では手書きのメモやイラストの中に自在にハイパーリンクを埋め込めるフォーマット「ハイパーイラスト」と、それらをナレッジとして扱える「手書きベースWiki」の開発を行った。
「手書きベースWiki」はハイパーリンクやWiki等の技術の組み合わせによって既存の手書きのメモ・イラスト問題点を克服し、新しい活用法を実現した。
今後は第\ref{chap:kosatsu}章で述べた問題点を受け、システムを改善していく。