\chapter{結論}
\label{chap:ketsuron}

本章では本研究を総括する。

\newpage

\section{研究の成果}
本研究では、ハイパーリンクを内蔵した次世代の手書きデータ記述形式「ハイパーイラスト」と、それをコンテンツとして扱うWikiシステム「手書きベースWiki」の提案を行った。

まず第\ref{chap:haikei}章において、既存の手書きメモ・イラストの問題点をテキストの進化と比較しながら分析した。
計算機上で手書きメモ・イラストを扱う既存システムの現状をとりあげ、計算機が普及した現在も根本的に解決されていないことを示した。

第\ref{chap:sekkei}章では、第\ref{chap:haikei}章で述べた手書きメモの問題点に対する解決方法として「手書きベースWiki」を提案した。
また、「手書きベースWiki」のプロトタイプとして実装した「DrawWiki」の機能を使い方と共に解説した。

第\ref{chap:zissou}章では、「DrawWiki」のアプリケーション構成と詳細な実装について述べた。

第\ref{chap:riyourei}章では、「DrawWiki」によって実現可能な利用例を紹介した。

第\ref{chap:kanren}章では、本研究に関連する研究を紹介し、それぞれのアプローチの特徴と問題点を分析し、本研究との比較検討を行った。

第\ref{chap:kosatsu}章では、ユーザーからのフィードバックや筆者による運用経験をもとに本研究の有効性と問題点を分析した。

\section{総括}
本研究では手書きのメモやイラストの中に自在にハイパーリンクを埋め込み、それらをナレッジとして扱える「手書きベースWiki」の開発を行った。
「手書きベースWiki」はハイパーリンクやWiki等の技術の組み合わせによって、参照や管理、再利用が難しいといった手書きのメモ・イラスト問題点を解決した。
また新しい活用法を実現した。
さらにテキストベースWikiに対しても優れた点があることを示した。
今後は第\ref{chap:kosatsu}章で述べた問題点を受け、システムを改善していく。