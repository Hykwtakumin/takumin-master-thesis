% ■ アブストラクトの出力 ■
%	◆書式:
%		begin{jabstract}〜end{jabstract}	:日本語のアブストラクト
%		begin{eabstract}〜end{eabstract}	:英語のアブストラクト
%		※ 不要ならばコマンドごと消せば出力されない。



% 日本語のアブストラクト
\begin{jabstract}

%手書きメモやイラスト等のグラフィカルなデータに自在にハイパーリンクを埋め込み、Wikiとして利用できるシステム「Draw Wiki」を提案する。
%手書きメモやイラストは広く浸透した表現手法であるものの、紙を前提としたフォーマットであるために(1)簡単に編集できない、(2)参照や管理が面倒、という問題が存在する。
%計算機上の手書きメモ作成・管理ツールや作図システムは広く利用されているものの、これらは紙のメモやイラストの利用形態を再現したに過ぎず、この問題を本質的に解決していない。
%「DrawWiki」とはWikiやSVG(XML)の組み合わせによって、簡単に手書きでメモやイラストを描きながら、自在にハイパーリンクを埋め込んだり、ハイパーリンクによって関連する他のメモやイラストを簡単に参照できるシステムである。
%これによって既存の手書きメモ・イラストの問題点が解決されるだけでなく、新しい活用法が提案できる。
%本論文では「Draw Wiki」の設計や実装、その応用例について述べ、研究の発展性について考察する。

 手書きメモやイラスト等のグラフィカルなデータに自在にハイパーリンクを埋め込み、それらをWikiとして利用できるシステム「手書きベースWiki」を提案する。
 手書きメモやイラストは広く浸透した情報の記録・表現手法であるものの、紙を前提としたフォーマットであるために参照や管理や再利用が難しいという問題が存在する。
 計算機上で手書きメモの作成や管理を行うツールは広く利用されているものの、これらは紙のメモやイラストの利用形態を再現したに過ぎず、この問題を本質的に解決していない。
 「手書きベースWiki」とはハイパーテキスト・ハイパーリンクやWiki等の技術の組み合わせによって、簡単に手書きでメモやイラストを描きながら、自在にハイパーリンクを埋め込んだり、
 ハイパーリンクによって関連する他のメモやイラストを簡単に参照できるシステムである。これによって既存の手書きメモ・イラストの問題点が解決されるだけでなく、新しい活用法が提案できる。
 本論文では「手書きベースWiki」の設計や実装、その応用例について述べ、研究の発展性について考察する。

\end{jabstract}


% 英語のアブストラクト
\begin{eabstract}

Eigo ga dekinai node Roma-ji de soreppoi hunniki wo daseruto iina.

Murippoi desu ne.

Write down your abstract here. Write down your abstract here. Write down your abstract here. Write down your abstract here. Write down your abstract here. Write down your abstract here.

 Write down your abstract here. Write down your abstract here. Write down your abstract here. Write down your abstract here. Write down your abstract here. Write down your abstract here. Write down your abstract here. Write down your abstract here. Write down your abstract here. Write down your abstract here. Write down your abstract here. Write down your abstract here.

Write down your abstract here. Write down your abstract here.

\end{eabstract}
