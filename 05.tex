\chapter{応用例}
\label{chap:ouyou}

本章では、手書きベースWikiによって実現可能な応用例について述べる。

\newpage

\section{個人用のメモ}
既存のツールにおいて手書きメモ・イラストは独立した画像ファイルとして扱われることが一般的であり、
ファイル名の工夫、保存先ディレクトリの指定、タグ付け等の工夫を行わない限り散逸してしまい、参照や管理が面倒という問題点があった。
Draw Wikiにおいては手書きメモ・イラストそのものにハイパーリンクが含まれており、さらに関連するハイパーイラストを一覧表示することも
できるため、一元管理することができる。

\section{チームにおけるプロトタイピング}
Draw Wikiで作成したハイパーイラストはURLを通じて他のユーザーもアクセス・編集することができる。
このため複数人でプロトタイピングのスケッチを行う場合等、共同編集で

\section{Webページとして公開する}
ハイパーイラストはそれ自体がハイパーリンクを内包したハイパーテキストである。
またDraw WikiではハイパーイラストにURLを割り振り、Webを通じてアクセス可能なようホスティングを行っている。
そのため手書きという簡単な操作のみでWebページを作り、公開することができる。

\subsubsection{手書きベースWikiの優位性}
テキストベースのWikiはテキストによって内容を記述することを前提としているが、
テキスト入力に起因する以下のような問題点がある。
\begin{enumerate}
    \item 効率的なテキスト入力はキーボード等の専用ハードウェアが必要であり、これらのデバイスを備えていないスマートフォンやタブレット等のハードウェアで便利に入力することができない
    \item キーボードの操作にはタッチタイピング等の技術に習熟している必要があるため、利用者にはある程度の技能が要求されてしまう。
    \item Wikiコンテンツの記述には専用の記法に倣う必要がある。例えばハイパーテキストを記述する上で代表的な言語であるHTMLでは、テキストへのハイパーリンクの埋め込みを以下のように定義する。
    \begin{lstlisting}[caption=htmlにおけるハイパーリンクの定義, label=htmlhyperlinking]
        <a href="https://example.com">HyperLink</a>
    \end{lstlisting}
    またHTMLへと変換できるプレーンテキストのフォーマットとして広く普及しているMarkdownでは、同様の構造を以下のように記述する。
    \begin{lstlisting}[caption=htmlにおけるハイパーリンクの定義, label=mdhyperlinking]
        [HyperLink]("https://example.com")
    \end{lstlisting}
    上記の記法とは別に独自の記法を採用しているWikiも存在するため、その各々の記法も利用者は網羅していなければならない。
\end{enumerate}

(1)についてはタッチパネルさえあれば手書き入力が可能であるため、スマートフォンやタブレット等のキーボードを持たないデバイスからでも利用することができる。
(2)に関しても、手書きはタイピングが登場する以前から存在する馴染み深いベーシックな入力方法であり、タッチタイピング等の技能を持たないが手書きはできる人でも利用することができる。
ハイパーリンクの定義も作成も手書きとシンプルな操作で完結するため(3)のような記法を覚える必要もない。

\section{まとめ}
本章では、本システムによって実現可能な応用例について述べた。
Wikiとハイパーリンクの組み合わせによって、既存の手書きメモ・イラストの問題点を解決することができた他、
テキストベースWikiに対しても優れた点があることがわかった。
本章で述べた応用例に限らず、様々な応用が可能と考えられる。