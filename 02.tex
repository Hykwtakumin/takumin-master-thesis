\chapter{研究背景}
\label{chap:haikei}

本章では手書きメモ・イラストを扱う既存のシステムの現状と、その問題点を整理する。

\newpage

\section{手書きのメモ・イラスト}
手書きによってメモやイラストを表現するのは、鉛筆等の筆記具と紙さえあればすぐに記録でき、また美麗な作品を描くことを目的としなければ、
特別な技量も要求されないため、現在でも汎用的な表現手法として広く普及している。
計算機の登場によりテキスト編集支援機能が充実したため、文章のみで完結する内容であれば手書きではなくタイピングによって記録するように置き換わったが、
アイデアのような文章のみで表現しづらい、構造を持った概念を表現する場合は文章と図を自在に混合させて配置できる手書きメモの方が適している。
また、テキストによってメモをとる場合はキーボードのような専用のハードウェアや、それらを使いこなすタッチタイピング等の技量が必要であるという問題点があるが、
手書きの場合は紙とペンで筆記することさえできれば良いため、テキスト入力に対してより多くの人々が利用できる手段であると言える。

\section{タッチパネル・ペンインターフェースの普及}
