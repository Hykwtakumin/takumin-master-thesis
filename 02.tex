\chapter{研究背景}
\label{chap:haikei}

本章では手書きメモ・イラストを扱う既存のシステムの現状と、その問題点を整理する。

\newpage

\section{手書きのメモ・イラスト}
手書きによってメモやイラストを表現するのは、鉛筆等の筆記具と紙さえあればすぐに記録でき、また美麗な作品を描くことを目的としなければ、
特別な技量も要求されないため、情報を記録・表現する手法として広く普及している。
計算機の登場によりテキスト編集支援機能が充実したため、文章のみで完結する内容であれば手書きではなくタイピングによって記録するように置き換わったが、
アイデアのような文章のみでは表現しづらい構造を持った概念を表現する場合は文字と図を自在に混合させて配置できる手書きメモの方が適している。
また、テキストによってメモをとる場合はキーボードのような専用のハードウェアや、それらを使いこなすタッチタイピング等の技量が必要であるという問題点があるが、
手書きの場合は紙やペン等の筆記具が扱えれば良いため、ハードウェアや技能を必要とするテキスト入力と比較してより多くの人々が利用できる手段であると言える。

\section{タッチ・ペンインターフェースの普及}

\begin{figure}[htbp] \begin{minipage}{0.5\hsize}
                         \begin{center} {\includegraphics[width=40mm]{images/ipadmini.jpg}}
                         \end{center} \caption{iPad}
\end{minipage} \begin{minipage}{0.5\hsize}
                   \begin{center} {\includegraphics[width=40mm]{images/surface.jpg}}
                   \end{center} \caption{Surface}
\end{minipage}
\end{figure}

かつてはノートやスケッチブック等の紙の上で記録されていた手書きメモだが、タッチパネルやスタイラスペン等のインターフェースを備えたデバイスの普及に伴い、
計算機上で手書きメモを取ることが一般化してきた。
手書きメモやイラストを計算機の上で描く場合、マウスやトラックパッド等のポインティングデバイスではなく、
ペンインターフェースが好ましいとされるが、iOS\footnote{https://www.apple.com/jp/ios/}やWindows\footnote{https://www.microsoft.com/ja-jp/windows}、
ChromeOS\footnote{https://www.google.com/chromebook/}等の主要なプラットフォームで
スタイラスペンを備えた機種が充実しているため手書きでメモやイラストを描く環境は充分に整っている。

\section{手書きデータを扱う既存のシステム}

\subsection{メモアプリケーション}


\subsubsection{iOSのメモ}

\begin{figure}[htbp]
    \begin{center}
        {\includegraphics[width=50mm]{images/applememo.png}} \end{center}
    \caption{iOSのメモ}
\end{figure}

AppleのiPadにはメモアプリケーションが標準でインストールされており、指やApple Pencilを用いて素早く手書きメモを取ることを可能で、
さらにUndo・Redo等の紙の上では実現出来なかった機能を備えている。一方で描いた手書きメモは画像として保存されるのみで、後から検索することができない。
本質的に手書きのメモを計算機上で再現したに過ぎず、故に紙の上でメモをしていた時代の不便さを引き継いでしまっている。

\subsubsection{Evernote}

\begin{figure}[htbp]
    \begin{center}
    {\includegraphics[width=50mm]{image.eps}} \end{center}
    \caption{Evernote}
\end{figure}

Evernote Corporationが開発するEvernoteは指やスタイラスペンで手書きのメモやスケッチを記録することができる。
手書きの文字を認識することで後からテキストによって検索する機能を備えているため、メモの参照のしづらさを解消している。
ただしこれは手書きメモをテキストに変換することで実現しているにすぎず、故に検索対象はテキストのみで、
図形や描いたもののシェイプ等のグラフィカルなデータから手書きメモを参照することはできない。
手書きメモ自体はラスタライズされた形式で保存されるため、紙の上で一枚絵としてメモやイラストを取っていた頃からさほど進歩していない。

\subsubsection{Google Keep}

\begin{figure}[htbp]
    \begin{center}
    {\includegraphics[width=50mm]{image.eps}} \end{center}
    \caption{GoogleKeep}
\end{figure}


Googleが開発するGoogle Keepも、Evernoteと同様にテキストによって検索することも可能である。
さらに画像や手書きメモ内にある文字を認識してテキストとして抽出する機能があるが、やはりこれも手書きメモをテキストに変換し、
テキストとして管理しているに過ぎず、手書きメモそのものは一枚の画像でしかないため紙の上のメモと変わりない。

\subsection{イラスト投稿・共有システム}

\subsubsection{Pixiv}

\begin{figure}[htbp]
    \begin{center}
    {\includegraphics[width=50mm]{image.eps}} \end{center}
    \caption{Pixiv}
\end{figure}

pixiv Inc.が開発するPixivでは、投稿したイラストに複数のタグを付加することができる。
また共通のタグを持つ他のイラストを関連イラストとして下部に表示する機能を備えているため、作者を横断して共通するテーマのイラストを参照することができる。

\subsubsection{ニコニ・コモンズ}

\begin{figure}[htbp]
    \begin{center}
    {\includegraphics[width=50mm]{image.eps}} \end{center}
    \caption{ニコニ・コモンズ}
\end{figure}

ドワンゴが開発するニコニコ動画の関連サービスであるニコニ・コモンズでは、イラストも含めた素材の親子関係を記述するコンテンツツリーという機能が実装されている。
これによりある作品の元になった作品や、ある作品を元にした他の作品を参照することができる。
ただし登録は子作品の投稿者が手動で行うという制約があるため、全てのコンテンツツリーが漏れなく記述されているわけではない。


\section{テキストの進化}
手書きのメモやイラストと同様に紙の上で表現されていたテキストの進化にも注目する。

\section{既存システムの問題点}

\section{まとめ}
手書きメモ・イラストは広く浸透した情報の記録手法であるものの、紙というフォーマットの制約によって使い勝手が制限されている。
一方でデジタル化された手書きメモ・イラストの作成や閲覧を支援するシステムが広く利用されているが、
これらは計算機上で手書きの利用形態を再現したに過ぎず、本質的な問題は解決されていない。
次章では上記のような問題点を解決し、これまでの手書きメモ・イラストの在り方にとらわれない次世代のフォーマット「ハイパーイラスト」と、
それらを容易に管理・再利用できるシステム「手書きベースWiki」を提案する。